\chapter{Concluzii și dezvoltare ulterioară}

Obiectivele acestei lucrări au fost proiectarea și programarea unui robot \mbox{explorator} într-un mod ce asigură decizii autonome în situații de risc și \mbox{garantează} un răspuns în timp real la comenzi și stimuli exteriori. Arhitectura modulară \mbox{propusă} s-a dovedit a fi potrivită acestui tip de problemă și conferă un grad înalt de flexibilitate. De asemenea, rezultatele experimentale au demonstrat că \mbox{integrarea} aplicației de control cu sistemul de operare în timp real FreeRTOS constituie o abordare nu numai elegantă, dar și eficientă. Aceste aspecte atestă succesul implementării funcțiilor esențiale ale unui robot explorator și stabilesc un punct solid de plecare pentru cercetările viitoare în domeniu. 

În continuare se au în vedere posibile dezvoltări ulterioare având ca obiect atât arhitectura hardware, cât și funcțiile implementate în software. O primă direcție de dezvoltare poate fi înlocuirea modulului Wi-Fi cu o altă piesă similară, ceea ce ar face posibilă transmisia unui flux de imagini dinspre robot către operator cu un număr suficient de cadre pe secundă, astfel încât această funcționalitate să fie cu adevărat utilă la explorarea spațiilor altfel inaccesibile.

De asemenea, robotul ar putea nu numai să transmită măsurători în timp real, dar și să înregistreze aceste date într-o memorie nevolatilă, de exemplu un card SD, oferind astfel posibilitatea realizării unor grafice și statistici ale variației parametrilor.

O funcționalitate nouă ce nu presupune modificarea hardware-ului existent poate fi dezvoltarea unei aplicații de comandă și control pe un telefon mobil, în scopul unei utilizări mai convenabile și mai flexibile.