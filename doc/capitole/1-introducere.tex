\chapter{Introducere}

Istoria roboților își are originile în Antichitatea clasică, fapt dovedit de scrierile matematicianului Heron din Alexandria, care stau mărturie existenței unor mecanisme automate \cite{heron} bazate pe principii mecanice. Fascinația pentru construcția unor astfel de mecanisme s-a manifestat și în perioada Renașterii italiene, prin schițele lui Leonardo da Vinci și mai târziu în timpul Revoluției Industriale.

Momentul de cotitură în acest domeniu îl constituie dezvoltarea electronicii și apariția circuitelor integrate digitale, în urma căruia a s-a cristalizat știința care se ocupă cu proiectarea, construcția și operarea roboților, precum și sistemele de control, reacție la stimuli exteriori și procesare a informației, sub numele de robotică.

Această ramură a tehnicii a cunoscut o ascensiune vizibilă în trecutul recent, o dată cu apariția Inteligenței Artificiale și a Internetului Obiectelor (\textit{engl. IoT}\footnote{Internet of Things}), conducând efectiv la integrarea circuitelor electronice cu sitemul de locomoție al robotului și posibilitatea de a conecta între ele mai multe astfel de sisteme autonome prin intermediul diverselor protocoale de comunicații.

În prezent, datorită progreselor rapide înregistrate în domeniu, roboții au ajuns să îndeplinească funcții din ce în ce mai diverse. Printre acestea se numără fabricarea de automobile, unde au ajuns să aibă o pondere tot mai mare, sub forma liniilor de producție complet automatizate, au ajuns să aibă o pondere tot mai mare pe parcursul ultimelor decenii. De asemenea, roboții au ajuns să fie folosiți extensiv și la împachetarea produselor alimentare, transportul mărfurilor în interiorul spațiilor de depozitare, recolta anumitor culturi de fructe, fabricarea cablajelor imprimate, precum și în exploatări miniere. Pe lângă tipurile de roboți industriali enumerate mai devreme, s-au dezvoltat roboți care pot duce independent la îndeplinire sarcini domestice ca tunsul ierbii sau aspirarea covoarelor și, nu în ultimul rând, roboți umanoizi, care au scopul de a imita comportamentul oamenilor sau părți din acesta, de exemplu mersul biped.

Spre deosebire de situațiile menționate anterior, unde roboții sunt folosiți cu scopul de a spori productivitatea unor activități ce pot fi efectuate de agenți umani, există împrejurări, care, prin natura lor, care impun folosirea unui robot. Poate cel mai bun exemplu este explorarea mediilor fizic inaccesibile oamenilor, aflate la mare distanță sau  având indici de risc care depășesc valori considerate admise, punând astfel în pericol o eventuală intervenție umană. Câteva exemple sunt interiorul conductelor, spațiul cosmic și mediile vulcanice sau puternic radioactive. Roboții special construiți cu scopul de a explora astfel de medii se încadrează în categoria roboților exploratori.


\section{Motivație}

Nivelul actual de dezvoltare a roboticii conduce la o popularitate în creștere a roboților exploratori în rândul comunității științifice și inginerești. Fie că este vorba despre roboți exploratori tereștri, sateliți artificiali sau drone, utilitățile acestora sunt nenumărate și se reflectă în numărul impresionant de cercetări privind explorarea spațiilor de orice natură și fascinația pe care o exercită asupra unei vaste categorii de oameni. Caracteristice acestor roboți sunt sistemele complicate necesare pentru a îndeplini chiar sarcini foarte simple, acoperind o multitudine de discipline diferite printre care se numără mecanica, electronica, telecomunicațiile și programarea.

La baza complexității roboților exploratori stă o varietate de constrângeri care se manifestă în moduri diferite în fiecare caz în parte. În particular, o dronă are nevoie de măsurători foarte precise ale înclinării și accelerației, precum și de un metodă eficientă de anulare a zgomotului, în timp ce calitatea cea mai importantă a unui robot destinat explorării spațiului cosmic este fiabilitatea. Un exemplu elocvent în acest caz este robotul \textit{Curiosity} al Agenției Spațiale Americane (\textit{engl. NASA}\footnote{National Aeronautics and Space Administration}), lansat în noiembrie 2011 pentru a explora planeta Marte.

\insfigs{curiosity.jpg}{Robotul explorator \textit{Curiosity} al NASA}{\href{https://www.jpl.nasa.gov/spaceimages/details.php?id=PIA09201}{www.jpl.nasa.gov} (15.05.2017)}{curiosity}{0.6}

La doar șase luni de la aterizarea acestuia, o eroare în memoria pilotului principal, combinată cu defectarea simultană a mecanismelor de siguranță a condus la un potențial dezastru \cite{curiosity} însemnând pierderea definitivă a întregii investiții de 2,5 miliarde de dolari americani.

Comună tuturor sistemelor mai sus amintite este constrângerea de timp real (\textit{engl. real-time constraint}) la care sunt supuse, definită garanția răspunsului la un stimul exterior înaintea unui termen limită prestabilit, indiferent de încărcarea sistemului. Soluția elegantă la această problemă este oferită de sistemele de operare în timp real (\textit{engl. RTOS}\footnote{Real-Time Operating System}).

Dezvoltarea în ultimul deceniu a sistemelor de operare în timp real se datorează în mare parte creșterii puterii de procesare a microcontrollerelor, rezultând în folosirea mai eficientă a timpului de calcul. Întrucât o schimbare de context\footnote{Schimbarea execuţiei unui proces cu un altul pe procesor} durează foarte puțin (ordinul microsecundelor), sistemele de operare actuale nu mai sunt constrânse de frecvența cu care apare, ceea ce conduce la un timp de răspuns îmbunătățit semnificativ.

În momentul de față există o multitudine de astfel de sisteme de operare, sub licență proprietară sau open source. Dintre acestea din urmă cel mai popular este \textit{FreeRTOS}, dezvoltat de \textit{Real Time Engineers Ltd.}, a cărui descriere detaliată se găsește în lucrarea de față la capitolul aferent. Acesta prezintă un număr mare de avantaje care îl recomandă, printre care faptul că e orientat pe evenimente, simplitatea interfeței de acces (\textit{engl. API}\footnote{Application Programming Interface}), documentația clară și abundentă, precum și suportul on-line de calitate.

Posibilitatea de a utiliza acest sistem de operare în timp real, flexibil și asincron, la baza unui robot explorator în scopul de a folosi eficient timpul de calcul și a obține un timp bun de răspuns la comenzi sau stimuli exteriori constituie motivația din spatele realizării acestui proiect.


\section{Scopul lucrării}

Lucrarea de față are drept scop proiectarea, asamblarea și programarea unui robot explorator operat de la distanță, destinat mediului terestru. Acesta trebuie să măsoare o varietate de parametri caracteristici mediului înconjurător precum temperatura, presiunea, umiditatea și intensitatea luminii ambientale și să transmită datele operatorului. În plus, este necesar ca acesta să reacționeze autonom în situații de risc ce îi pun în pericol integritatea. Găsirea unei modalități eficiente de a garanta că atât răspunsul la comenzile de deplasare, cât și transferul continuu al datelor se petrec în timp real, face, de asemenea, obiectul acestei lucrări.


\section{Stadiul actual}

Un robot destinat explorării terestre este dotat cu roți sau, cel mai frecvent, montat pe un șasiu cu șenile, ceea ce îi conferă mobilitate și stabilitate crescută pe terenuri accidentate și facilitează ocolirea sau depășirea eventualelor obstacole. Un exemplu în acest sens este robotul \textit{Urbie Rover} al NASA.

\insfigs{urbie.jpg}{Robotul urban \textit{Urbie Rover} al NASA}{\href{https://www.jpl.nasa.gov/news/news.php?feature=485}{www.jpl.nasa.gov}  (16.05.2017)}{urbie}{0.6}

Acesta servește în special la explorarea mediilor urbane, caracterizate de obstacole dificil de trecut ca serii de trepte sau borduri. Este prevăzut cu șenile, wireless Ethernet, receptor GPS diferențial, busolă digitală, LIDAR\footnote{Light Imaging, Detection, And Ranging}, o cameră de luat vederi omnidirecțională, o pereche de camere video pentru vedere binoculară și două procesoare Intel Pentium PC-104. Arhitectura sa este una modulară, unul din procesoare fiind rezervat pentru navigație, iar celălalt pentru procesarea de imagini. Aceste echipamente îi oferă posibilitatea de a îndeplini autonom o varietate de sarcini, precum localizarea în spațiu, urcarea treptelor, evitarea obstacolelor și reconstituirea și retraversarea traseului parcurs.

Cu certitudine se poate spune că, în această categorie, cei mai utili sunt roboții proiectați să opereze în spații de dimensiuni reduse, spre exemplu în interiorul unor conducte de ventilație, cu scopul de a localiza posibile defecte sau chiar în scopul cercetărilor arheologice. Elocvent în acest sens este cazul robotului explorator \textit{Djedi}, dezvoltat de comunitatea internațională de egiptologi în colaborare cu Universitatea din Leeds.

\insfigs{djedi.png}{Robotul explorator \textit{Djedi}}{\href{http://archive.archaeology.org/1109/trenches/djedi_project_robot_pyramid_khufu_palenque.html}{archive.archaeology.org} (20.05.2017)}{djedi}{0.6}

Un caz aparte, \textit{Djedi} a fost special proiectat și construit în scopul de a explora puțurile de ventilație din Marea Piramidă a lui Kheops \cite{djedi}, un exemplu excelent de problemă pentru care utilizarea unui robot este unica soluție, explorarea acestui mediu fiind cu desăvârșire imposibil de realizat folosind orice altă metodă.


\section{Sisteme embedded}

Sistemul expus în lucrarea de față intră în categoria sistemelor încorporate sau \textit{embedded} în timp real, dedicate doar rezolvării unui anumit tip de sarcini și atingerii simultane a unor țeluri cuantificabile. Este necesar ca la proiectarea unui astfel de sistem să se țină seama de aceste țeluri, din care cel mai important este garanția răspunsului în timp real, dar printre care se numără și consumul de energie și, nu în ultimul rând, costul. Cerințele specifice conduc inevitabil la compromisuri diferite între performanță și consumul de energie sau accentul predominant pus pe hardware sau software.

Contrastând puternic cu calculatoarele de uz general, unde nu se cunoaște în prealabil modul în care vor fi folosite și ale căror arhitecturi trebuie să permită folosirea lor în feluri cât mai diverse, în cazul unui sistem embedded atât proiectarea hardware cât și software se realizează simultan. Prin urmare, o problemă întâmpinată poate fi rezolvată în hardware, software sau o combinație între cele două. Aceste abordări implică compromisuri diferite, deci oferă un spațiu mai mare de manevră și posibilitatea de a obține o soluție calitativ mai bună, mai fiabilă și flexibilă, fără însă de a prezenta flexibilitate inutil de mare în detrimentul îndeplinirii cerințelor. \cite{wolf}

Un sistem embedded operează în timp real atâta timp cât execută procesele critice într-un timp acceptabil. Definiția flexibilă a acestui interval îl încadrează ca parte a cerințelor comportamentale ale sistemului și nu a celor funcționale. Este obligatoriu ca aceste cerințe să fie măsurabile sau cuantificabile în mod obiectiv, dar pot fi reduse la un model de administrare a resurselor ce poate fi statică sau dinamică, aflată în sarcina unui programator, respectiv a unui sistem de operare. Administrarea resurselor în timp real are însă asociat un cost, deoarece este imposibil de obținut gradul de încredere într-un timp rezonabil de răspuns doar prin redimensionarea hardware-ului. În contextul interacțiunii cu alte dispozitive, un sistem embedded are nevoie atât de capabilități hardware adecvate, cât și de eficiență în administrarea software a resurselor. Mediul este complex și în continuă schimbare și acest lucru se reflectă în numărul și ponderea interacțiunilor. Importanța maximă în acest context aparține evaluării corecte a timpului fizic, necesară pentru a corela evenimentele cu momentele exacte ale producerii lor, nu cea logică, asupra căreia sistemele au în mod automat control. Prin urmare, caracteristica esențială a sistemelor embedded în timp real devine inevitabil găsirea unui compromis între performanță și timpul definit ca acceptabil. \cite{kraeling}

